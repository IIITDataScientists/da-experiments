% LaTeX Template for Project Report, Version 2.0
% (Abstracted from a Major Project Report at CSED, NIT Calicut but can be
% modified easily to use for other reports also.)
%
% Released under Creative Commons Attribution license (CC-BY)
% Info: http://creativecommons.org/licenses/by/3.0/
%
% Created by: Kartik Singhal
% BTech CSE Batch of 2009-13
% NIT Calicut
% Contact Info: kartiksinghal@gmail.com
%
% It is advisable to learn the basics of LaTeX before using this template.
% A good resource to start with is http://en.wikibooks.org/wiki/LaTeX/
%
% All template fields are marked with a pair of angular brackets e.g. <title here>
% except for the ones defining citation names in ref.tex.
%
% Empty space after chapter/section/subsection titles can be used to insert text.
%
% Just compile this file using pdflatex after making all required changes.

\documentclass[11pt,a4paper]{report}
\usepackage{array}
\usepackage[pdftex]{graphicx} %for embedding images
\usepackage{url} %for proper url entries
\usepackage[bookmarks, colorlinks=false, pdfborder={0 0 0}, pdftitle={SSLC Data Analysis}, pdfauthor={Arjun S Bharadwaj, Apoorwa Mishra, Ankit Shah}, pdfsubject={SSLC Data}, pdfkeywords={sslc data, analytics}]{hyperref} %for creating links in the pdf version and other additional pdf attributes, no effect on the printed document
%\usepackage[final]{pdfpages} %for embedding another pdf, remove if not required

\newcolumntype{L}[1]{>{\raggedright\let\newline\\\arraybackslash\hspace{0pt}}m{#1}}
\newcolumntype{C}[1]{>{\centering\let\newline\\\arraybackslash\hspace{0pt}}m{#1}}
\newcolumntype{R}[1]{>{\raggedleft\let\newline\\\arraybackslash\hspace{0pt}}m{#1}}

\begin{document}
\renewcommand\bibname{References} %Renames "Bibliography" to "References" on ref page
\renewcommand*{\chaptername}{Experiment}
%include other pages
\begin{titlepage}

\begin{center}

\textup{\small {\bf DS707 Data Analytics} \\ Project Report}\\[0.2in]

% Title
\Large \textbf {SSLC Data Analysis}\\[0.5in]       
        \vspace{.2in}

       {\bf Masters of Technology \\in\\ Information Technology}\\[0.5in]

% Submitted by
\normalsize Submitted by \\
\begin{table}[h]
\centering
\begin{tabular}{lr}\hline \\
Roll No & Names of Students \\ \\ \hline
\\
MT2013025 & Apoorwa Mishra \\ \\
MT2013026 & Arjun S Bharadwaj \\ \\
MT2013140 & Ankit Shah \\ \\ \hline 
\end{tabular}
\end{table}

\vspace{.1in}
Under the guidance of\\
{\textbf{Prof. Chandrashekar R}}\\[0.2in]

\vfill

% Bottom of the page
\includegraphics[width=0.18\textwidth]{./iiitb-logo}\\[0.1in]
\normalsize
\textsc{International Institute of Information Technology}\\
Bangalore, Karnataka, India -- 560 100 \\
\vspace{0.2cm}
Fall Semester 2014

\end{center}

\end{titlepage}

\pagenumbering{roman} %numbering before main content starts
\tableofcontents

\newpage
\pagenumbering{arabic} %reset numbering to normal for the main content
\chapter{Discretizaion + Classification}
Consider the marks information: \\
L1\textunderscore MARKS, L2\textunderscore MARKS, L3\textunderscore MARKS, S1\textunderscore MARKS, S2\textunderscore MARKS, S3\textunderscore MARKS. Consider TOTAL\textunderscore MARKS as the dependent variable
\begin{itemize}
	\item
	Objective:
	\begin{itemize}
		\item
		Discretize subject marks into discrete attributes S (use NRC\textunderscore CLASS as domain)
		\item
		Build a classification model based on S for NRC\textunderscore CLASS class variable
	\end{itemize}
	
	\item
	Procedure followed: 
	\begin{itemize}
		\item
		Step 1
		\item
		Step 2
	\end{itemize}
	
	\item
	Results Obtained:
	\begin{itemize}
		\item
		Result 1
		\item
		Result 2
	\end{itemize}
	
	\item
	Conclusion:
	\begin{itemize}
		\item
		Conclusion 1
		\item
		Conclusion 2
	\end{itemize}
\end{itemize}
\chapter{Regression + Classification}
Consider the marks information: \\
L1\textunderscore MARKS, L2\textunderscore MARKS, L3\textunderscore MARKS, S1\textunderscore MARKS, S2\textunderscore MARKS, S3\textunderscore MARKS. Consider TOTAL\textunderscore MARKS as the dependent variable
\begin{itemize}
	\item
	Objective:
	\begin{itemize}
		\item
		Determine the least number of attributes S that give the best possible regression equation (least error)
		\item
		Build a classification model based on S for NRC\textunderscore CLASS class variable
	\end{itemize}
	
	\item
	Procedure followed: 
	\begin{itemize}
		\item
		Data is loaded and cleansed by removing invalid and missing rows.
		\item
		Regression analysis is performed on the data by using the marks data.
		\item
		Synergy/Interaction effect of all the marks are obsereved and the combination of marks having least p-value is chosen for classification.
		\item
		Marks are rounded off for improving the speed of classification.
		\item
		Classification is performed on the data based on the class variable combination having least p-value.
		\item
		The results of confusion matrices are compared.
	\end{itemize}
	
	\item
	Results Obtained:
	\begin{itemize}
		\item
		All Subjects are used for classification:
		\begin{itemize}
			\item
			Accuracy: 90.2\%
			\item
			95\% CI: (0.8962, 0.9076)
		\end{itemize}
		
		\item
		Best case: \\	
		L1\textunderscore MARKS, L2\textunderscore MARKS, S2\textunderscore MARKS, S3\textunderscore MARKS (p-value = 0.0732) are used for classification:
		\begin{itemize}
			\item
			Accuracy: 84.03\%
			\item
			95\% CI: (0.8332, 0.8472)
		\end{itemize}
		
		\item
		Worst case: \\
		L3\textunderscore MARKS, S1\textunderscore MARKS (p-value = 0.94523) are used for classification:
		\begin{itemize}
			\item
			Accuracy: 69.46\%
			\item
			95\% CI: (0.6857, 0.7033)
		\end{itemize}
	\end{itemize}
	
	\item
	Conclusion:
	\begin{itemize}
		\item
		Taking all the subjects marks for classification gives the highest accuracy.
		\item
		Taking the combination of the subjects having low p-value offers the next highest accuracy for classification.
		\item
		Conversely, the combination of subjects having highest p-value gives the least accuracy.
	\end{itemize}
\end{itemize}
\chapter{Clustering + Association Rules}
Consider the marks information: \\
L1\textunderscore MARKS, L2\textunderscore MARKS, L3\textunderscore MARKS, S1\textunderscore MARKS, S2\textunderscore MARKS, S3\textunderscore MARKS.
\begin{itemize}
	\item
	Objective:
	\begin{itemize}
		\item
		Create clusters based on M
		\item
		Characterize each cluster individually by creating association rules (Use discretized subject marks as ITEMS)
	\end{itemize}
	
	\item
	Procedure followed: 
	\begin{itemize}
		\item
		Step 1
		\item
		Step 2
	\end{itemize}
	
	\item
	Results Obtained:
	\begin{itemize}
		\item
		Result 1
		\item
		Result 2
	\end{itemize}
	
	\item
	Conclusion:
	\begin{itemize}
		\item
		Conclusion 1
		\item
		Conclusion 2
	\end{itemize}
\end{itemize}
\chapter{Regression + Classification}
Consider the marks information: \\
L1\textunderscore MARKS, L2\textunderscore MARKS, L3\textunderscore MARKS, S1\textunderscore MARKS, S2\textunderscore MARKS, S3\textunderscore MARKS. Consider TOTAL\textunderscore MARKS as the dependent variable
\begin{itemize}
	\item
	Objective:
	\begin{itemize}
		\item
		Determine the least number of attributes S that give the best possible regression equation (least error)
		\item
		Build a classification model based on S for NRC\textunderscore CLASS class variable
	\end{itemize}
	
	\item
	Procedure followed: 
	\begin{itemize}
		\item
		Step 1
		\item
		Step 2
	\end{itemize}
	
	\item
	Results Obtained:
	\begin{itemize}
		\item
		Result 1
		\item
		Result 2
	\end{itemize}
	
	\item
	Conclusion:
	\begin{itemize}
		\item
		Conclusion 1
		\item
		Conclusion 2
	\end{itemize}
\end{itemize}
\chapter{Urban / Rural Characterization}
What are the characteristics of students from urban and rural areas, respectively? For antecedent, try with demographic info (SCHOOL\textunderscore TYPE, URBAN\textunderscore RURAL, NRC\textunderscore CASTE\textunderscore CODE, NRC\textunderscore GENDER\textunderscore CODE, NRC\textunderscore MEDIUM, NRC\textunderscore PHYSICAL\textunderscore CONDITION, CANDIDATE\textunderscore TYPE)
Also try with subject performance in the antecedent
\begin{itemize}
	\item
	Objective: \\
	Identify association rules with URBAN\textunderscore RURAL in the consequent of the rule	
	
	\item
	Procedure followed: 
	\begin{itemize}
		\item
		Data is loaded and cleansed by removing invalid and missing rows.
		\item
		The values of all the columns are factored so that it's suitable for association rules analysis.
		\item
		Apriori algorithm is run on the data by forcing URBAN\textunderscore RURAL=Rural rule in consequent.
		\item
		Apriori algorithm is run on the data by forcing URBAN\textunderscore RURAL=Urban rule in consequent.
		\item
		The rules generated with high confidence and lift are compared for both the cases.
	\end{itemize}
	
	\item
	Results Obtained: \\
	For URBAN\textunderscore RURAL = Urban in the consequent, the following were the results:
		\begin{center}
		    \begin{tabular}{| L{8.25cm} | c | c | c |}
		    \hline
				Antecedant & Support & Confidence & Lift \\ \hline
				SCHOOL\textunderscore TYPE = Unaided, NRC\textunderscore MEDIUM = English & 0.1305375 & 0.8029823 & 1.883977 \\ 	\hline
				SCHOOL\textunderscore TYPE = Unaided, NRC\textunderscore MEDIUM = English, NRC\textunderscore PHYSICAL\textunderscore CONDITION = Normal & 0.1297800 & 0.8025108 & 1.882871 \\ 	\hline
				SCHOOL\textunderscore TYPE = Unaided, NRC\textunderscore MEDIUM = English, NRC\textunderscore CASTE\textunderscore CODE = General & 0.1130235 & 0.8002575 & 1.877584 \\ 	\hline
			\end{tabular}
		\end{center}
		
	For URBAN\textunderscore RURAL = Rural in consequent, the following were the top three results:
		\begin{center}
		    \begin{tabular}{ | L{8.25cm} | c | c | c |}
		    \hline
				Antecedant & Support & Confidence & Lift \\ \hline
				SCHOOL\textunderscore TYPE = Government, NRC\textunderscore GENDER\textunderscore CODE=Boy, NRC\textunderscore MEDIUM = Kannada, CANDIDATE\textunderscore TYPE=Regular Fresher, L1\textunderscore RESULT = Pass, L2\textunderscore RESULT = Pass, S2\textunderscore RESULT = Pass, S3\textunderscore RESULT = Pass & 0.1018423 & 0.8611325 & 1.500797 \\ 	\hline		

				SCHOOL\textunderscore TYPE = Government, NRC\textunderscore GENDER\textunderscore CODE = Boy, NRC\textunderscore MEDIUM = Kannada, CANDIDATE\textunderscore TYPE = Regular Fresher, L1\textunderscore RESULT = Pass, L2\textunderscore RESULT = Pass, L3\textunderscore RESULT = Pass, S1\textunderscore RESULT = Pass, S3\textunderscore RESULT = Pass & 0.1015393 & 0.8609969 & 1.500561 \\ 	\hline
				
				SCHOOL\textunderscore TYPE = Government, NRC\textunderscore GENDER\textunderscore CODE = Boy, NRC\textunderscore MEDIUM = Kannada, CANDIDATE\textunderscore TYPE = Regular Fresher, L1\textunderscore RESULT = Pass, L2\textunderscore RESULT = Pass, L3\textunderscore RESULT = Pass, S1\textunderscore RESULT = Pass, S2\textunderscore RESULT = Pass & 0.1012969 & 0.8609323 & 1.500448 \\ 	\hline						
			\end{tabular}
		\end{center}
	
	\item
	Conclusion:
	\begin{itemize}
		\item
		Students in Urban area mainly belong to Unaided English medium schools.
		\item
		Students in Rural area are mainly boys who belong to Government Kannada medium schools.
	\end{itemize}
\end{itemize}
\chapter{Performance characteristics}
Use D (Distinction) and FAIL in the consequent of association rule. For antecedent, try with demographic info (SCHOOL\textunderscore TYPE, URBAN\textunderscore RURAL, NRC\textunderscore CASTE\textunderscore CODE, NRC\textunderscore GENDER\textunderscore CODE, NRC\textunderscore MEDIUM, NRC\textunderscore PHYSICAL\textunderscore CONDITION, CANDIDATE\textunderscore TYPE). Also try with subject performance in the antecedent.

\begin{itemize}
	\item
	Objective:
	\begin{itemize}
		\item
		What properties characterize high and poor performers?		
	\end{itemize}
	
	\item
	Procedure followed: 
	\begin{itemize}
		\item
		Step 1
		\item
		Step 2
	\end{itemize}
	
	\item
	Results Obtained:
	\begin{itemize}
		\item
		Result 1
		\item
		Result 2
	\end{itemize}
	
	\item
	Conclusion:
	\begin{itemize}
		\item
		Conclusion 1
		\item
		Conclusion 2
	\end{itemize}
\end{itemize}
\chapter{Decision tree vis-a-vis A-rules}
Can we use decision trees to validate association rules or vice-versa?
\begin{itemize}
	\item
	Objective:
	\begin{itemize}
		\item
		Note the strongest rules that have been found
		\item
		Then induce a decision tree based on those attributes
		\item
		Validate the decision tree using standard metrics
	\end{itemize}
	
	\item
	Procedure followed: 
	\begin{itemize}
		\item
		Step 1
		\item
		Step 2
	\end{itemize}
	
	\item
	Results Obtained:
	\begin{itemize}
		\item
		Result 1
		\item
		Result 2
	\end{itemize}
	
	\item
	Conclusion:
	\begin{itemize}
		\item
		Conclusion 1
		\item
		Conclusion 2
	\end{itemize}
\end{itemize}
\chapter{Cross - cluster analysis}
Create a clustering C1 of the overall population. Then create a clustering C2 of partitioned population separately (e.g., gender-based)
\begin{itemize}
	\item
	Objective:
	\begin{itemize}
		\item
		Compare cluster C1 with C2.
		\item
		Are the characteristics same? Show it by statistical analysis.
	\end{itemize}
	
	\item
	Procedure followed: 
	\begin{itemize}
		\item
		The data file is loaded, the invalid data is removed and the L1\textunderscore MARKS is normalised to 100.		
		\item
		The data is split into three parts: Overall population data, Boys data, Girls data.
		\item
		The value of k = 5 is selected and k-means is run of all the three datasets.
	\end{itemize}
	
	\item
	Results Obtained:
	\begin{itemize}
		\item
		The range of mean of marks for all the subjects across the three datasets are as follows:
		\begin{center}
		    \begin{tabular}{ | C{2.5cm} | C{1.5cm} | C{1.5cm} | C{1.5cm} | C{1.5cm} | C{1.5cm} |}
		    \hline
				Dataset \textbackslash  Cluster & 1 & 2 & 3 & 4 & 5 \\ \hline
				Overall population data & 54.6 - 75.79 & 43.38 - 62.23 & 73.36 - 87.19 & 19.39 - 27.20 & 35.05 - 43.52 \\ 	\hline		
				Boys data & 72.66 - 85.43 & 16.86 - 24.90 & 32.93 - 41.34 & 42.26 - 58.07 & 54.01 - 72.40 \\ 	\hline		
				Girls data & 21.11 - 29.24 & 74.6 - 88.71 & 36.53 - 47.63 & 44.61 - 67.27 & 55.74 - 79.22 \\ 	\hline									
			\end{tabular}
		\end{center}
		
		\item
		When the cluster are analysed with NRC\textunderscore CLASS, the following matrix is obtained:
		
		\begin{itemize}
			\item
			Overall Population data: 
			\begin{center}
		    \begin{tabular}{ | c | c | c | c | c | c |}
		    \hline
			   	 Class \textbackslash Cluster & 	Distinction  & Fail  & First & Pass  & Second \\ 	\hline
				  1      &     0  &  0   & 6065 &  9   &  677  \\ 	\hline		
				  2      &     0  & 129  &   95 & 2992 &  4967 \\ 	\hline		
				  3      &  1356  &  0   & 2927 &  0   &  0 	  \\ 	\hline		
				  4      &    0   & 4164 &  0   &   0  &  0    \\ 	\hline		
				  5      &     0  & 2502 &   0  & 6120 &  0    \\ 	\hline		
				\end{tabular}
			\end{center}
			
			\item
			Boys data: 
			\begin{center}
		    \begin{tabular}{ | c | c | c | c | c | c |}
		    \hline
			   	 Class \textbackslash Cluster & 	Distinction  & Fail  & First & Pass  & Second \\ 	\hline				     
						  1     &    597 &   0  & 1485 &   0  &    0  \\ 	\hline
						  2     &      0 & 2139 &    0 &   0  &    0  \\ 	\hline
						  3     &      0 & 2048 &    0 & 2692 &    0  \\ 	\hline
						  4     &      0 &  96  &   0  & 2399 &  1961 \\ 	\hline
						  5     &      0 &   2  & 2646 &   8  &  842	 \\ 	\hline	
				\end{tabular}
			\end{center}
						
			\item
			Girls data: 
			\begin{center}
		    \begin{tabular}{ | c | c | c | c | c | c |}
		    \hline
			   	 Class \textbackslash Cluster & 	Distinction  & Fail  & First & Pass  & Second \\ 	\hline
						  1     &      0 & 1821 &    0 &   0  &    0 \\ 	\hline
						  2     &    759 &   0  & 1327 &   0  &    0 \\ 	\hline
						  3     &      0 & 651  &   0  & 3296 &    0 \\ 	\hline
						  4     &      0 &  38  & 470  & 725  & 2761 \\ 	\hline
						  5     &      0 &   0  & 3159 &   1  &   80 \\ 	\hline
				\end{tabular}
			\end{center}
		\end{itemize}
	\end{itemize}
	
	\item
	Conclusion:\\
	The characteristics are slightly different, however the pattern across clusters are similar.
	\begin{itemize}
		\item
		The average score range is lesser in boys data compared to overall population data and girls data indicationg girls performing better in every cluster.
		\item
		The width of the range of average scores is more is boys data and overall population data than girls data indicating the standard deviation is low for boys in every cluster.
		\item
		In the boys data, the fail and pass class are almost equally distributed in the clusters.
		\item
		In the girls data, the pass class has more distribution than fail class in the clusters.
	\end{itemize}
\end{itemize}
\chapter{Score Prediction - Additional Experiment}
Prediction of the score using regression equation.
\begin{itemize}
	\item
	Objective:
	\begin{itemize}
		\item
		Predict the total marks of the candidate from the regression equation.
	\end{itemize}
	
	\item
	Procedure followed: 
	\begin{itemize}
		\item
		The data file is loaded and the invalid data is removed.
		\item
		The regression formulation is done by specifying the class variables and the predictors.
		\item
		The data is passed to the lm function and the equation is obtained based on the training data.
		\item
		The equation is now used to predict TOTAL\textunderscore MARKS of the test data.
	\end{itemize}
	
	\item
	Results Obtained:\\
	
		The topper data is given to the model for prediction. The following result is obtained:
		\begin{center}
		    \begin{tabular}{ | c | c | c |}
		    \hline
			   	 Actual Score & 	Predicted Score  & Actual Score - Predicted Score \\ 	\hline
				610	& 610.2218	& -0.2217667 \\ 	\hline
				610	& 610.2105	& -0.2105289 \\ 	\hline
				612	& 612.2313	& -0.2313323 \\ 	\hline
				611	& 611.2340	& -0.2339893 \\ 	\hline
				613	& 613.2204	& -0.2203704 \\ 	\hline
				619	& 619.2363	& -0.2362730 \\ 	\hline
				611	& 611.2160	& -0.2160255 \\ 	\hline
				612	& 612.2188	& -0.2187575 \\ 	\hline
				620	& 464.1491	& 155.8508606 \\ \hline
				615	& 447.1857	& 167.8142784 \\ \hline
	
			\end{tabular}
		\end{center}
	
	\item
	Conclusion:\\	
	\begin{itemize}
		\item
		The generated regression equation is an accurate equation and it can be seen with the predicted data.
		\item
		The predicted data in the last two observations indicate a large difference, this shows that the total marks was tampered artificially.
		\item
		The co-efficients of all the intercepts are almost equal to 1. This makes it highly accurate.		
	\end{itemize}
\end{itemize}
\end{document}
