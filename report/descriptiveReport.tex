\section*{Descriptive Report}
This section describes the dataset that was used for performing various experiments using R programming language. Some of the notable features of the dataset are as follows:
\begin{itemize}
	\item
	The 	data set has 33,002 observations of 38 variables.
	\item
	The data set has 32,003 valid observations that is suitable for analytics.
	\item
	The maximum marks scored by the student is 620 and the minimum marks is 6.
	\item
	The mean of total marks is 321.6.
	\item
	Every subject has at least one person with maximum score, but there is no one with maximum score in all the subjects.
	\item
	Just 4.23\% of students have scored distinction, 21.23\% have failed. First and Second class have about 28\% of students each.
	\item
	About 40\% of students belong to government schools, 30\% belong to aided schools and remaining to unaided schools.
	\item
	About 57.41\% belong to rural area and rest to the urban area.
	\item
	70\% of candidates belong to general category. 17\% belong to SC, rest belong to ST and Category 1.
	\item
	About 53\% of candidates are boys, rest are girls.
	\item
	69\% of students belong to Kannada medium, 25\% belong to English. Rest belong to other medium schools.
	\item
	99.78\% of candidates are normal and others have disability.
	\item
	90.26\% of candidates are regular freshers.
	\item
	In all the subjects, the pass percentages of individual subjects are in the range of 84\% - 92\%. But still, more than 22\% have failed in overall result! This indicates the candidates who have failed have to focus more on few subjects.
\end{itemize}