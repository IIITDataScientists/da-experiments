\chapter{Score Prediction - Additional Experiment}
Prediction of the score using regression equation.
\begin{itemize}
	\item
	Objective:
	\begin{itemize}
		\item
		Predict the total marks of the candidate from the regression equation.
	\end{itemize}
	
	\item
	Procedure followed: 
	\begin{itemize}
		\item
		The data file is loaded and the invalid data is removed.
		\item
		The regression formulation is done by specifying the class variables and the predictors.
		\item
		The data is passed to the lm function and the equation is obtained based on the training data.
		\item
		The equation is now used to predict TOTAL\textunderscore MARKS of the test data.
	\end{itemize}
	
	\item
	Results Obtained:\\
	
		The topper data is given to the model for prediction. The following result is obtained:
		\begin{center}
		    \begin{tabular}{ | c | c | c |}
		    \hline
			   	 Actual Score & 	Predicted Score  & Actual Score - Predicted Score \\ 	\hline
				610	& 610.2218	& -0.2217667 \\ 	\hline
				610	& 610.2105	& -0.2105289 \\ 	\hline
				612	& 612.2313	& -0.2313323 \\ 	\hline
				611	& 611.2340	& -0.2339893 \\ 	\hline
				613	& 613.2204	& -0.2203704 \\ 	\hline
				619	& 619.2363	& -0.2362730 \\ 	\hline
				611	& 611.2160	& -0.2160255 \\ 	\hline
				612	& 612.2188	& -0.2187575 \\ 	\hline
				620	& 464.1491	& 155.8508606 \\ \hline
				615	& 447.1857	& 167.8142784 \\ \hline
	
			\end{tabular}
		\end{center}
	
	\item
	Conclusion:\\	
	\begin{itemize}
		\item
		The generated regression equation is an accurate equation and it can be seen with the predicted data.
		\item
		The predicted data in the last two observations indicate a large difference, this shows that the total marks was tampered artificially.
		\item
		The co-efficients of all the intercepts are almost equal to 1. This makes it highly accurate.		
	\end{itemize}
\end{itemize}