\chapter{Urban / Rural Characterization}
What are the characteristics of students from urban and rural areas, respectively? For antecedent, try with demographic info (SCHOOL\textunderscore TYPE, URBAN\textunderscore RURAL, NRC\textunderscore CASTE\textunderscore CODE, NRC\textunderscore GENDER\textunderscore CODE, NRC\textunderscore MEDIUM, NRC\textunderscore PHYSICAL\textunderscore CONDITION, CANDIDATE\textunderscore TYPE)
Also try with subject performance in the antecedent
\begin{itemize}
	\item
	Objective: \\
	Identify association rules with URBAN\textunderscore RURAL in the consequent of the rule	
	
	\item
	Procedure followed: 
	\begin{itemize}
		\item
		Data is loaded and cleansed by removing invalid and missing rows.
		\item
		The values of all the columns are factored so that it's suitable for association rules analysis.
		\item
		Apriori algorithm is run on the data by forcing URBAN\textunderscore RURAL=Rural rule in consequent.
		\item
		Apriori algorithm is run on the data by forcing URBAN\textunderscore RURAL=Urban rule in consequent.
		\item
		The rules generated with high confidence and lift are compared for both the cases.
	\end{itemize}
	
	\item
	Results Obtained: \\
	For URBAN\textunderscore RURAL = Urban in the consequent, the following were the results:
		\begin{center}
		    \begin{tabular}{| L{8.25cm} | c | c | c |}
		    \hline
				Antecedant & Support & Confidence & Lift \\ \hline
				SCHOOL\textunderscore TYPE = Unaided, NRC\textunderscore MEDIUM = English & 0.1305375 & 0.8029823 & 1.883977 \\ 	\hline
				SCHOOL\textunderscore TYPE = Unaided, NRC\textunderscore MEDIUM = English, NRC\textunderscore PHYSICAL\textunderscore CONDITION = Normal & 0.1297800 & 0.8025108 & 1.882871 \\ 	\hline
				SCHOOL\textunderscore TYPE = Unaided, NRC\textunderscore MEDIUM = English, NRC\textunderscore CASTE\textunderscore CODE = General & 0.1130235 & 0.8002575 & 1.877584 \\ 	\hline
			\end{tabular}
		\end{center}
		
	For URBAN\textunderscore RURAL = Rural in consequent, the following were the top three results:
		\begin{center}
		    \begin{tabular}{ | L{8.25cm} | c | c | c |}
		    \hline
				Antecedant & Support & Confidence & Lift \\ \hline
				SCHOOL\textunderscore TYPE = Government, NRC\textunderscore GENDER\textunderscore CODE=Boy, NRC\textunderscore MEDIUM = Kannada, CANDIDATE\textunderscore TYPE=Regular Fresher, L1\textunderscore RESULT = Pass, L2\textunderscore RESULT = Pass, S2\textunderscore RESULT = Pass, S3\textunderscore RESULT = Pass & 0.1018423 & 0.8611325 & 1.500797 \\ 	\hline		

				SCHOOL\textunderscore TYPE = Government, NRC\textunderscore GENDER\textunderscore CODE = Boy, NRC\textunderscore MEDIUM = Kannada, CANDIDATE\textunderscore TYPE = Regular Fresher, L1\textunderscore RESULT = Pass, L2\textunderscore RESULT = Pass, L3\textunderscore RESULT = Pass, S1\textunderscore RESULT = Pass, S3\textunderscore RESULT = Pass & 0.1015393 & 0.8609969 & 1.500561 \\ 	\hline
				
				SCHOOL\textunderscore TYPE = Government, NRC\textunderscore GENDER\textunderscore CODE = Boy, NRC\textunderscore MEDIUM = Kannada, CANDIDATE\textunderscore TYPE = Regular Fresher, L1\textunderscore RESULT = Pass, L2\textunderscore RESULT = Pass, L3\textunderscore RESULT = Pass, S1\textunderscore RESULT = Pass, S2\textunderscore RESULT = Pass & 0.1012969 & 0.8609323 & 1.500448 \\ 	\hline						
			\end{tabular}
		\end{center}
	
	\item
	Conclusion:
	\begin{itemize}
		\item
		Students in Urban area mainly belong to Unaided English medium schools.
		\item
		Students in Rural area are mainly boys who belong to Government Kannada medium schools.
	\end{itemize}
\end{itemize}