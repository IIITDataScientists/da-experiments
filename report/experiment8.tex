\chapter{Cross - cluster analysis}
Create a clustering C1 of the overall population. Then create a clustering C2 of partitioned population separately (e.g., gender-based)
\begin{itemize}
	\item
	Objective:
	\begin{itemize}
		\item
		Compare cluster C1 with C2.
		\item
		Are the characteristics same? Show it by statistical analysis.
	\end{itemize}
	
	\item
	Procedure followed: 
	\begin{itemize}
		\item
		The data file is loaded, the invalid data is removed and the L1\textunderscore MARKS is normalised to 100.		
		\item
		The data is split into three parts: Overall population data, Boys data, Girls data.
		\item
		The value of k = 5 is selected and k-means is run of all the three datasets.
	\end{itemize}
	
	\item
	Results Obtained:
	\begin{itemize}
		\item
		The range of mean of marks for all the subjects across the three datasets are as follows:
		\begin{center}
		    \begin{tabular}{ | C{2.5cm} | C{1.5cm} | C{1.5cm} | C{1.5cm} | C{1.5cm} | C{1.5cm} |}
		    \hline
				Dataset \textbackslash  Cluster & 1 & 2 & 3 & 4 & 5 \\ \hline
				Overall population data & 54.6 - 75.79 & 43.38 - 62.23 & 73.36 - 87.19 & 19.39 - 27.20 & 35.05 - 43.52 \\ 	\hline		
				Boys data & 72.66 - 85.43 & 16.86 - 24.90 & 32.93 - 41.34 & 42.26 - 58.07 & 54.01 - 72.40 \\ 	\hline		
				Girls data & 21.11 - 29.24 & 74.6 - 88.71 & 36.53 - 47.63 & 44.61 - 67.27 & 55.74 - 79.22 \\ 	\hline									
			\end{tabular}
		\end{center}
		
		\item
		When the cluster are analysed with NRC\textunderscore CLASS, the following matrix is obtained:
		
		\begin{itemize}
			\item
			Overall Population data: 
			\begin{center}
		    \begin{tabular}{ | c | c | c | c | c | c |}
		    \hline
			   	 Class \textbackslash Cluster & 	Distinction  & Fail  & First & Pass  & Second \\ 	\hline
				  1      &     0  &  0   & 6065 &  9   &  677  \\ 	\hline		
				  2      &     0  & 129  &   95 & 2992 &  4967 \\ 	\hline		
				  3      &  1356  &  0   & 2927 &  0   &  0 	  \\ 	\hline		
				  4      &    0   & 4164 &  0   &   0  &  0    \\ 	\hline		
				  5      &     0  & 2502 &   0  & 6120 &  0    \\ 	\hline		
				\end{tabular}
			\end{center}
			
			\item
			Boys data: 
			\begin{center}
		    \begin{tabular}{ | c | c | c | c | c | c |}
		    \hline
			   	 Class \textbackslash Cluster & 	Distinction  & Fail  & First & Pass  & Second \\ 	\hline				     
						  1     &    597 &   0  & 1485 &   0  &    0  \\ 	\hline
						  2     &      0 & 2139 &    0 &   0  &    0  \\ 	\hline
						  3     &      0 & 2048 &    0 & 2692 &    0  \\ 	\hline
						  4     &      0 &  96  &   0  & 2399 &  1961 \\ 	\hline
						  5     &      0 &   2  & 2646 &   8  &  842	 \\ 	\hline	
				\end{tabular}
			\end{center}
						
			\item
			Girls data: 
			\begin{center}
		    \begin{tabular}{ | c | c | c | c | c | c |}
		    \hline
			   	 Class \textbackslash Cluster & 	Distinction  & Fail  & First & Pass  & Second \\ 	\hline
						  1     &      0 & 1821 &    0 &   0  &    0 \\ 	\hline
						  2     &    759 &   0  & 1327 &   0  &    0 \\ 	\hline
						  3     &      0 & 651  &   0  & 3296 &    0 \\ 	\hline
						  4     &      0 &  38  & 470  & 725  & 2761 \\ 	\hline
						  5     &      0 &   0  & 3159 &   1  &   80 \\ 	\hline
				\end{tabular}
			\end{center}
		\end{itemize}
	\end{itemize}
	
	\item
	Conclusion:\\
	The characteristics are slightly different, however the pattern across clusters are similar.
	\begin{itemize}
		\item
		The average score range is lesser in boys data compared to overall population data and girls data indicationg girls performing better in every cluster.
		\item
		The width of the range of average scores is more is boys data and overall population data than girls data indicating the standard deviation is low for boys in every cluster.
		\item
		In the boys data, the fail and pass class are almost equally distributed in the clusters.
		\item
		In the girls data, the pass class has more distribution than fail class in the clusters.
	\end{itemize}
\end{itemize}